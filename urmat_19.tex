\documentclass[11pt,a4paper]{article}
\usepackage[utf8x]{inputenc}
\usepackage[T1]{fontenc}
\usepackage[russian]{babel}
\usepackage{ucs}
\usepackage{amsmath}
\usepackage{amsfonts}
\usepackage{amssymb}
\usepackage{graphicx}
\usepackage{fullpage}
\usepackage[left=0.1cm,right=0.1cm,top=0.1cm,bottom=0cm,bindingoffset=0.1cm]{geometry}

\begin{document}
    \section*{Билет 19. Решение внутренней краевой задачи Дирихле для уравнения Лапласа в круге. Формула Пуассона}
    Задача Дирихле в круге:
    \[ \left\{\begin{aligned}
    & \Delta u=0, \\ & u(r,\varphi)|_{r=a}=f(\varphi)
    \end{aligned}\right. \]
    Или в полярных координатах:
    \[ \left\{\begin{aligned}
    & \frac{1}{r} \frac{\partial}{\partial r}\left (r\frac{\partial u}{\partial r}\right)+\frac{1}{r^2}\frac{\partial^2 u}{\partial^2 \varphi}=0, \\ & u|_{r=a}=f(\varphi), 0\leq \varphi<2\pi, 0\leq r\leq a
    \end{aligned}\right. \]
    \textbf{Решение:}
    Метод разделения переменных: $u(r,\varphi)=R(r)\Phi(\varphi)$
    Подставим в задачу Дирихле в полярных координатах:
    \begin{equation}\label{eq:5}
    \frac{r\frac{\partial }{\partial r}\left(r\frac{\partial R}{\partial r} 
    \right)}{R(r)}=-\frac{\Phi''(\varphi)}{\Phi(\varphi)}=\lambda=const
    \end{equation}
    Получим:
    \begin{equation}\label{eq:1}
    r\frac{\partial }{\partial r}\left(r\frac{\partial R}{\partial r} 
    \right)-\lambda R(r)=0
    \end{equation}
    Уравнение Эйлера:
    \begin{equation}
    r^2R''+rR'-\lambda R=0
    \end{equation}
    Частное решение в виде $R=r^m, m=const$.
    \begin{center}
    $r^2(m-1)mr^{m-2}+rmr^{m-1}-\lambda r^m=0$\\
    $m=\pm \lambda^2 (\lambda>0)$ или $\lambda=0$
    \end{center}
    Если $\lambda=0$, то $R(r)=C_0\ln{r}+C_1$.
    Решение должно быть ограничено в центре круга при $r=0$, поэтому из двух найденных решений берем $r^{\lambda^2}=r^n$ и $u(r,\varphi)$ должна быть непрерывной и конечной в круге.
    \begin{center}
    $\Phi''+\lambda\Phi=0, \Phi(\varphi)=\Phi(\varphi+2\pi)$ \\
    $\Phi(\varphi)=A_n\cos{n\varphi}+B_n\sin{n\varphi}$
    \end{center}
    Все частные решения: $u_n(r,\varphi)=r^n(A_n\cos{n\varphi}+B_n\sin{n\varphi}), n=0,1...$
    \begin{equation}
    u(r,\varphi)=\sum_{n=0}^\infty r^n(A_n\cos{n\varphi}+B_n\sin{n\varphi})
    \end{equation}
    Найдем $A_n$ и $B_n$ разложение $f(\varphi)$ в $(0,2\pi)$ в ряде Фурье.
    \begin{equation}
    f(\varphi)= \frac{\alpha_0}{2}+\sum_{n=1}^\infty r^n(\alpha_n\cos{n\varphi}+\beta_n\sin{n\varphi})
    \end{equation}
    $\alpha_0=\frac{1}{\pi}\int\limits_0^{2\pi} f(\varphi)\,d\varphi$,
    $\alpha_n=\frac{1}{\pi}\int\limits_0^{2\pi} f(\varphi)\cos{n\varphi}\,d\varphi, \quad n=0,1..$ \\
    $\beta_n=\frac{1}{\pi}\int\limits_0^{2\pi} f(\varphi)\sin{n\varphi}\,d\varphi, \quad n=0,1..$ \\
    $f(\varphi)= \sum_{n=0}^\infty a^n(A_n\cos{n\varphi}+B_n\sin{n\varphi}) $ \\
    Тогда получаем:
    $A_0=\frac{\alpha_0}{2}, A_n=\frac{\alpha_n}{a^n}, B_n=\frac{\beta_n}{a^n}$
    \begin{equation}
    u(r,\varphi)=\frac{\alpha_0}{2}+\sum_{n=1}^\infty \left(\frac{r}{a}\right)^n(\alpha_n\cos{n\varphi}+\beta_n\sin{n\varphi})
    \end{equation}
    Подставим выражение для коэффициентов Фурье в формулу $(6)$, тогда получим:
    \begin{center}
    $$u(r,\varphi)=\frac{1}{\pi}\int\limits_0^{2\pi} f(\alpha)\left(\frac{1}{2}+\sum_{n=1}^\infty \left(\frac{r}{a}\right)^n(\cos{n\varphi}\cos{n\alpha}+\sin{n\varphi}\sin{n\alpha})\right) \,d\alpha = \frac{1}{\pi}\int\limits_0^{2\pi} f(\alpha)\left(\frac{1}{2}+\sum_{n=1}^\infty \left(\frac{r}{a}\right)^n(\cos{n(\varphi-\alpha)} \right) \,d\alpha$$\\
    \newpage
    $$\left\{\cos{n(\varphi-\alpha)}=
    \frac{e^{in(\varphi-\alpha)}+e^{-in(\varphi-\alpha)}}{2}, q=\frac{r}{a}<1 \right\};\,\frac{1}{2}\left(1+\sum_{n=1}^\infty \left[{(qe^{i(\varphi-\alpha)})}^n+{(qe^{-i(\varphi-\alpha)})}^n \right] \right)=$$ \\
    $$=\frac{1}{2}\left(1+\frac{qe^{i(\varphi-\alpha)}}{1-qe^{i(\varphi-\alpha)}}+\frac{qe^{-i(\varphi-\alpha)}}{1-qe^{-i(\varphi-\alpha)}} \right)=\frac{1}{2}\frac{1-q^2}{1-2q\cos{(\varphi-\alpha)}+q^2}=\frac{1}{2}\frac{a^2-r^2}{a^2-2ar\cos{(\varphi-\alpha)}+r^2}$$
    \end{center}
    \begin{equation}
    u(r,\varphi)=\frac{1}{2\pi}\int\limits_0^{2\pi} \frac{f(\alpha)(a^2-r^2)}{a^2-2ar\cos{(\varphi-\alpha)}+r^2}\,d\alpha, \quad (r<a)
    \end{equation}
    \section*{2. Теоремы единственности и устойчивости решения задачи Коши для уравнения колебания}
    Задача Коши для однородного уравения колебаний: $-\infty<x<\infty$, $t>0$.
    \[ \left\{\begin{aligned}
    &u_{tt}=a^2u_{xx}, \\ & u(x,0)=\varphi(x), u_t(x,0)=\psi(x)
    \end{aligned}\right. \]
    \textbf{Теорема.}
    Пусть $u(x)$ дважды непрерывно дифференцируема, а $\psi(x)$ непрерывно дифференцируема на бесконечной прямой. Тогда решение задачи Коши существует, единственно и определяется формулой Даламбера:\\ $u(x,t)=\frac{\varphi(x-at)+\varphi(x+at)}{2}+\frac{1}{2a}\int\limits_{x-at}^{x+at} \psi(\alpha) \,d\alpha$.
    \\
    \textbf{Теорема.}
    Пусть начальные функции $\varphi_s(x)$ и $\psi_s(x)$ $(s=1,2)$ двух задач Коши, удовлетворяют условиям :\\
    $|\varphi_1(x)-\varphi_2(x)|\leq \epsilon$ , $x\in (-\infty ; \infty)$
    и $\int\limits_a^b |\psi_1(z)-\psi_2(z)|^2 \,dz \leq \epsilon^2(b-a)^2 $,
    тогда выполняется для решений этих задач с $t\in[0,T]$ $|u_1(x,t)-u_2(x,t)|\leq\epsilon(1+T)$ (устойчивость).
    
\end{document}    