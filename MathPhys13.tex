\documentclass[11pt,a4paper]{article}
\usepackage[utf8x]{inputenc}
\usepackage[T1]{fontenc}
\usepackage[russian]{babel}
\usepackage{ucs}
\usepackage{amsmath}
\usepackage{amsfonts}
\usepackage{amssymb}
\usepackage{graphicx}
\usepackage{fullpage}
\usepackage{indentfirst}
\usepackage[left=0.1cm,right=0.1cm,top=0.1cm,bottom=0.1cm,bindingoffset=0cm]{geometry}

\title{Билет 13}
\date{}

\begin{document}
    % \maketitle 
\section*{Билет 13}
\section{Решение задачи Коши для уравнения теплопроводности на полубесконечной прямой с нулевым краевым условием первого рода.}

    \noindent
    При распределении температуры влияние одного из концов стержня несущественно, можно им пренебречь. 
    Рассмотрим \textbf{первую краевую задачу на полупрямой:} 
        \begin{equation}\label{eq1}
            \begin{cases}
                u_t = a^2u_{xx}, & x > 0,\;t > 0 \\ 
                u(x,0) = \varphi(x), & x > 0 \\
                u(0,t) = 0, & t > 0
            \end{cases} 
                \varphi(0) = 0
        \end{equation}
    Доопределим нечетным образом \(\varphi(x)\):
        \begin{equation}\label{eq2}
            \Phi = \begin{cases}
                        \varphi(x), & x \ge 0 \\
                        -\varphi(-x), & x < 0
                    \end{cases}
        \end{equation}
    Рассмотрим \textbf{задачу Коши:}
        \begin{equation}\label{eq3}
            \begin{cases}
                U_t = a^2U_{xx}, & -\infty < x < +\infty,\;t > 0 \\ 
                U(x,0) = \Phi(x), & -\infty < x < +\infty \\
                U(0,t) = 0, & t \ge 0
            \end{cases}
        \end{equation}
    Известно: 
        \begin{equation}\label{eq4}
            U(x) =  \int_{-\infty}^{+\infty}\frac{1}{\sqrt{4\pi a^2 t}} \exp\{{-\frac{(x-s^2)}{4a^2t}}\}\Phi(s)\,ds - \textit{интеграл Пуассона}  
        \end{equation}
    Покажем, что \(u(x,t) = U(x,t)\) при \(x \ge 0\), является решением \eqref{eq1}. Проверим выполнение граничного условия:
        \begin{equation}\label{eq5}
            u(0, t) = U(0, t) = \int_{-\infty}^{+\infty}\frac{1}{\sqrt{4\pi a^2 t}} \exp\{{-\frac{s^2}{4a^2t}}\}\Phi(s)\,ds = 
            \left\{
            \begin{aligned}
                & \text{под интегралом произведение} \\
                & \text{\;четной и нечетной функций =} \\
                & \text{\;\;\;\;\;\;= нечетной функции }                
             \end{aligned}
            \right\} = 0
        \end{equation}
    Получим формулу для решения:
        \begin{gather}\nonumber
            u(x,t) = \int_{-\infty}^{+\infty}\frac{1}{\sqrt{4\pi a^2 t}} \exp\{{-\frac{(x-s^2)}{4a^2t}}\}\Phi(s)\,ds = \\
            = \frac{1}{\sqrt{4\pi a^2 t}} \cdot 
            \left( 
                \int_{-\infty}^{0} \exp\{{-\frac{(x-s^2)}{4a^2t}}\}-\varphi(-s)\,ds + 
                \int_{0}^{+\infty} \exp\{{-\frac{(x-s^2)}{4a^2t}}\}\varphi(s)\,ds
            \right) = \\\nonumber
            =  \frac{1}{\sqrt{4\pi a^2 t}} \cdot \label{eq6}
            \left( 
                -\int_{0}^{+\infty}\exp\{{-\frac{(x+s^2)}{4a^2t}}\varphi(s)\,ds + 
                \int_{0}^{+\infty}\exp\{{-\frac{(x-s^2)}{4a^2t}}\varphi(s)\,ds 
            \right) = \\\nonumber
            =  \frac{1}{\sqrt{4\pi a^2 t}}\,
            \int_{0}^{+\infty}
            \left[
                \exp\{{-\frac{(x-s^2)}{4a^2t}}\varphi(s) -
                \exp\{{-\frac{(x+s^2)}{4a^2t}}\varphi(s) 
            \right] * \varphi(s)\,ds                
        \end{gather}
    В итоге, с учетом \eqref{eq5} и \eqref{eq6}, имеем:
        \begin{equation}\label{eq7}
            \begin{cases}
                u_t = a^2u_{xx}, & x > 0,\;t > 0 \\ 
                u(x,0) = \varphi(x), & x > 0 \\
                u(0,t) = 0, & t > 0
            \end{cases} 
        \end{equation} 
    Значит \(u(x,t)\) - есть \textbf{решение первой краевой задачи на полупрямой.}        

\section{Задача Гурса. Сведение к эквивалентной системе интегральных уравнений.}
\setcounter{equation}{0}

    \noindent
    Классическое уравнение в частных производных 2-го порядка:
    \begin{equation}\label{eq1}
        a_{11}(x, y)u_{xx} + 2a_{12}(x, y)u_{xy} + a_{22}(x, y)u_{yy} = F(x, y, u, u_x, u_y)
    \end{equation}
    Поставим ему в однозначное соответствие обыкновенное дифференциальное уравнение:
    \begin{equation}\label{eq2}
        a_{11}(dy)^2 - 2a_{12}dxdy + a_{22}(dx)^2 = 0
    \end{equation}
    Тогда функции (кривые), являющиеся решением \eqref{eq2}, называются \textbf{характеристиками уравнения} \eqref{eq1}. \\
    
    \noindent
    Рассмотрим задачу с нелинейным уравнением гиперболического типа:
    \begin{equation}\label{eq3}
        \left\{
        \begin{aligned}
            u_{xy}(x,y) &= a(x,y)u_x + b(x,y)u_y + f(x,y,u(x,y)),\;0 < x < l_1,\;0 < y < l_2 \\
            u(x,0) &= \varphi(x),\; 0 \le x \le l_1 \\
            u(0,y) &= \psi(y),\; 0 \le y \le l_2 
        \end{aligned}
        \right.
        \textbf{\;\;\;\;\;- задача Гурса}
    \end{equation}
    Для первого уравнения \eqref{eq3} характеристиками будут функции удовлетворяющие: \(dx dy = 0\).\\
    Получим семейство прямых вида: \(x = const, y = const\). \\
    Таким образом, наша u(x,y) задается данными на характеристиках \(x = 0, y = 0\).\\
    
    \noindent
    \textbf{Эквивалентная система интегральных уравнений} \\
    Пусть функция \(u(x, y)\) — решение задачи Гурса \eqref{eq3}. Тогда, интегрируя уравнение 
    сначала по \(y\), потом по \(x\):
    \begin{equation}\label{eq4}
        u(x,y) = \psi(y) + \varphi(x) − \varphi(0) + \int \limits_0^x \int \limits_0^y
        \left[
            a(\xi,\eta)u_x(\xi,\eta) + b(\xi,\eta)u_y(\xi,\eta)
        \right]\,d\eta d\xi + \int \limits_0^x \int \limits_0^y
        f(\xi,\eta,u(\xi, \eta))\,d\eta d\xi.
    \end{equation}
    \begin{equation}\label{eq5}
        u_x(x,y) = \varphi\,'(x) + \int \limits_0^y
        \left[
            a(x,\eta)u_x(x,\eta) + b(x,\eta)u_y(x,\eta)
        \right]\,d\eta + \int \limits_0^y
        f(x,\eta,u(x, \eta))\,d\eta.
    \end{equation}
    \begin{equation}\label{eq6}
        u_y(x,y) = \psi\,'(y) + \int \limits_0^x
        \left[
            a(\xi,y)u_x(\xi,y) + b(\xi,y)u_y(\xi,y)
        \right]\, d\xi + \int \limits_0^x
        f(\xi,y,u(\xi, y))\, d\xi.
    \end{equation}
    Получим \eqref{eq4}, \eqref{eq5} и \eqref{eq6} - система эквивалентных интегральных уравнений задаче \eqref{eq3}.
\end{document}